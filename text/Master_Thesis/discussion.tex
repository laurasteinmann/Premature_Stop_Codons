% !TeX root = main.tex
\chapter{Discussion}
Our analysis of the premature stop codons in the 1,135 \textit{A. thaliana} population gives new insights into the effect of natural selection on them. We find premature stop codons distributed all over the 665 accessions of \textit{A. thaliana}. Remarkably not just a few but many hundreds of premature stop codon mutation occur for each accession.  Moreover these are distributed in the population at different positions and even if they emerged in the same gene different position are observed in the population of \textit{A. thaliana}. We analyse the gene expression level of premature stop codons and find three categories. The category in which premature stop codons influence an increase of gene expression in the knock-out accessions are especially interesting for further analysis and so far do not have a biological interpretation. We further gained interesting insights into the distribution of premature stop codons in genes relative to their insertion site and found an accumulation of premature stop codons at the corners for very short and nearly complete remaining proteins. We identify a strong natural selection influencing the premature stop codon in the middle region, where medium remaining proteins result. Our analysis for the comparison of the protein length and mRNA expression based analysis shows, that the gene expression decrease is not entirely correlated to the length of the resulting protein. Even by considering proteins, which just remain with \SI{30}{\percent} of their original length we find genes with a decrease in gene expression, which need to have another influence factor. Our analysis of interactions between premature stop codons shows that previous hypothesis of the single occurrence of premature stop codons can not be confirmed in our dataset. We identify pairs of premature stop codons which significantly are linked together on a population scale and hypothesize from these finding a mechanism. This mechanism results in adaptation to an environment by gaining premature stop codons even more by gaining pairs of premature stop codons which positively force the generation of new gene regulatory networks by preventing the functionality of the former ones.  

Already in 2012 MacArthur et al. \cite{macarthur2012} showed, that an average human carries about 100 loss-of-function variants in his genome, without building severe disorders. These results match with our findings of the observed distribution of premature stop codon in \textit{A. thaliana} even if we see a slight increase of these numbers since we obtain on average several hundred premature stop codons for an \textit{A. thaliana} plant. It was shown by Sharma et al. 2018 \cite{Sharma2018}, that loss-of-function variants like premature stop codons are not only resulting in a reduced fitness but also shape phenotypic diversity in evolution. They base these insights on mammalian adaptation but we can match these with our findings of the accumulation of premature stop codons and their natural variation in the population of \textit{A. thaliana}. Our analysis shows that this process is not only true for mammalian adaptationbut that once can observe the same effects in the plant \textit{A. thaliana}. Former studies on budding yeast further revealed that the loss of a specific part of a genetic network can result in an opening of an alternative evolutionary path in adaptation (Helsen, 2020 \cite{Helsen2020}). This strengthens our findings of significant pairs of premature stop codons, which are linked by evolution and occur together more often than you would expect by chance and which switch of gene regulatory networks. As we see in our coocurrence analysis, we find specific pairs of premature stop codons, which influence their presence and absence in relation to each other. We also know through decades of studying \textit{A. thaliana} premature stop codons can have different influence in different ecotypes, which we can also observe in our dataset. We can see that although some premature stop codons are linked together, they emerge only in some of the accessions and not in the total population. This means that for one ecotype and its related gene-regulatory network the gene loss can be compensated. It can even mean, that the adaptation to its environmental conditions is enhanced for this ecotype while this loss-of-function variant results in a negative fitness effect for other ecotypes. The different effect on fitness  of the emergence of loss-of-function variant in different individuals was shown by Rojas Echenique et al. 2019 \cite{Echenique2019} in microbial cells. It was also shown by Helsen et al. 2020 \cite{Helsen2020}, that loosing highly connected genes in a gene-regulatory network allows the increase of phenotypic diversity after adaptation. Interpreting our results by considering these finding in budding yeast we can assume that the linked premature stop codon pairs found in our population of \textit{A. thaliana} are representing such highly linked genes, which are also called master regulators. Summarizing these different studies and also finding these different processes in a representative \textit{A. thaliana} population emphasizes the possibility of gaining a more detailed look at adaptation and evolution by using the new \grqq omics\grqq{}-technologies. By sequencing genomes as well as whole transcriptomes on a population scale we can observe evolutionary effects on large amounts of individuals, which is only possible through the technological progress in all of these methods. This is even further shown in a study by Monroe et al. 2022 \cite{Monroe2022}, which demonstrates that our former evolutionary theory might be disproven. The axioms of evolutionary theory are based on the hypothesis that mutations occur completely random throughout the genome and that mutations are selected with respect to their consequences only after their emergence. But as these researchers found differences of mutations in the occurrence throughout the genome and falsified this axiom of evolution by testing this assumption in a large survey of \textit{A. thaliana} plants. In summary this shows how much we can still learn about evolution and that we can still find new insights in the natural variation of population on how evolution shaped our existing natural habitat.


\chapter{Outlook}
Although our analysis gives a first insight into the influence of premature stop codons influence on the adaptation of plants, the project is not finished by this master thesis. As already explained in the method section, the premature stop codon cooccurrence needs to be interpreted in relation to other control scenarios. The first control is the application of GWAS to the already prepared phenotypes. With this analysis we can remedy our bonferroni corrected statistical cooccurrence of premature stop codon of another effect: the population structure. With a hypergeometric test it is not possible to correct for population effects but by applying GWAS to our interaction pairs we can correct for this effect and therefore even strengthen the significance of these cooccurring pairs of premature stop codons. 

An additional control is needed to compare the number of pairs of premature stop codons in the genomic context. As explained in the methods section we plan to use synonymous mutations as a control group since they represent mutations, which do not show any effect on protein functionality. Preliminary results for this are presented at the end of \autoref{sec:Results_Control}. By evaluating these results we will be able to see how different these two mutation classes behave. We can further use it to define a noise background anc correlate a rate of false positives identified in our cooccurrence analysis. In the future, we can even extend these control to the group of non-synonymous mutations, which do have a medium severity and can compare these different mutation classes in this aspect of cooccurrence. 

As indicated in the results section we want to continue the project by analysing gene-regulatory-networks. We plan to set up gene-regulatory networks based on our coocurrence analysis and study the effect of these pairs of premature stop codons. We hope to gain more insight into which biochemical pathways are involved and if there can be more detailed predictions on how these networks behave in different accessions. 

Finally we detected an interesting category of premature stop codons. Premature stop codons, that influence the gene expression in a positive regulation since they increase their transcription. This is not easy to explain in the biological context. When studying these specific premature stop codons further, it would be very interesting to analyse their differences in relation to cooccurrence and gene-regulatory network features. 
