% !TeX root = main.tex
\chapter{Material}
In this chapter we will look at the necessary computing infrastructure and code base that we developed for studying the selection on loss-of-function variants. At the end of this chapter we will introduce the \textit{A. thaliana} dataset we used for our analysis.
\section{Computational Material and Resources}
We start with outlining the computing infrastructure that is necessary to perform our analysis.

All analysis are performed on the institutes own high-performance computer cluster, with access to a compute node with 80 cores and 512 GB RAM. While parts of the analysis require less computational resources, the maximum capacity of this compute node is needed for calculation on the basic unfiltered dataset, especially due to the large RAM requirement. To recapitulate the complete analysis access, a high-performance computer node is therefore necessary. 

In the following we elaborate on the availability and dependencies of the code base used for these analysis. To study how premature stop codons get distributed in the genetic pool and shape the adaptation in plants we followed a completely bioinformatical based approach. We developed the statistical analysis and data analysis in two programming languages. While we mainly use python, R was a necessary additional language to complete the workflow. They are set up on an linux based system (Debian). All necessary information to recapitulate the analysis can be found in a github repository at \url{https://github.com/laurasteinmann/Premature_Stop_Codons.git}.

Although at the moment there are no strict dependency on software versions I will list the used packages and their versions since there is always the possibility of semantic or functional changes in future versions. For the Python analysis part we use python (3.10) and the common packages for dealing with numerical data and big data analysis like pandas (1.4) and  numpy (1.22). For calculating the statistical tests we use scipy (1.8.0). For visualizing our results we use the matplotlib package (3.5), the seaborn package (0.11) and the matplotlib\_venn package (0.11). For dealing with genomic data and filtering it we use the R language with version 4.1.3 and the vcfR library (1.12).
\section{\textit{A. thaliana} dataset of 1001 Genomes Project}
The flowering plant \textit{Arabidopsis thaliana} is the standard reference plant for many disciplines in biology. For research questions in physiological, cellular, and molecular processes it is one of the most studied model organisms (Koornneef 2010 \cite{Koornneef2010}). \textit{A. thaliana} especially gained importance for studying genes and determining their function since it was the first plant genome sequence in 2000 (Arabidopsis Genomes Initiative \cite{AGI2000}). With a relatively small genome of 125 megabase pairs that include about 25 000 genes, \textit{A. thaliana} emphasized its importance as a model organism. This reference genome is based on the col-0 ecotype of \textit{A. thaliana} and its annotation and quality was improved with the technological progress since then.

\textit{A. thaliana} is also a promoted model organism for studying natural variation since \textit{A. thaliana} is not known for agricultural usage or as a crop plant. This means that in \textit{A. thaliana} we can still observe natural selection and not the selection of human interests \cite{Koornneef2010}. Since the generation and the publishing of genomic sequences of 1, 135 ecotypes (accessions) of \textit{A. thaliana} in the 1001 Genomes project (1001 Genomes Consortium\cite{1001Genomes2016})  we have an ideal dataset to perform the analysis of adaptation in plants at population scale. These 1,135 accessions come from a worldwide hierarchical collection, which includes ecotypes from Sweden, the Iberian Peninsula as well as from North America and Central Asia. The accessions are a collection of naturally inbred lines that represent individuals under diverse ecological conditions. By sequencing with Illumina short read sequencing genome-wide polymorphisms for these genomes were characterized. In total a collection 10,707,430 SNPs were identified. Based on this genomic information we can try to quantify genomic variation in a representative sample of accessions.

As an additional resource the 1001 Genomes Consortium provides 727 accessions with completely sequenced transcriptomes (Kawakatsu 2016\cite{Kawakatsu2016}). These accessions have an average expression of 18,000 genes and allow us to combine the information of genomic and transcriptomic data. Unfortunately these 727 accessions overlap not completely with the 1,135 accessions. A joint analysis of both datasets can therefore only be based on the information of the overlapping 665 accessions. 
\chapter{Methods}
In this chapter we focus on the methods behind our analysis. We take a look at how the necessary statistical methods work and how we prepare the available information to extract the data needed to understand the selection in plant populations further. 
\section{Overlap of Genomic and Transcriptomic Dataset}
Before we start our analysis we decided to filter the dataset we use for the overlap of the genomic and transcriptomic dataset. Therefore we selected the 665 accessions of \textit{A. thaliana}, for which pieces of both information are available.
\section{Distribution of premature stop codons}
For analyzing the distribution of premature stop codons we extracted the information of all premature stop codons in the 665 population of \textit{A. thaliana}. We connect each stop codon to the related gene. Based on this knowledge we summarized first the number of premature stop codons in each of the accessions and second the number of premature stop codons occurring in each gene. \autoref{fig:Distribution_Premature_Stop_Codons_all} visualizes our results.
\section{Classification of premature stop codons based on their gene expression differences} 
\label{sec:Methods_Approach_Gene_Expression}
Since the results show in \autoref{sec:Distribution_Results} we decided to generate a high confidential dataset. These dataset should consist of premature stop codons, that effect protein functionality. To build such a high confidential dataset we split into two separate approaches. In the first approach we classify genes based on their gene expression and in the second approach we classify based on the relative length of the resulting protein. 

For the first approach we start with some filtering steps before classifying accessions into two subsets, the wildtypes and the knock-out-accessions, for each premature stop codon. By doing this we can compare these subsets by calculating a t-test to separate premature stop codons with a significant gene expression change. In the following these steps are explained in a detailed way to understand the process of generating the first high confidential dataset. 
\subsection*{Classification of accessions}
We start to build our confidential dataset by building two groups of accessions for each premature stop codon. All accessions, which do not have a premature stop codon at a particular position are classified as wildtype accessions. Accessions, which do have a premature stop codon there are classified as knock out accessions (k.o.-accessions) and accessions, which have a missing value at that position, are separated from the analysis. Missing values indicate sequencing errors or short deletions occurring at that position and do not allow a classification of these accessions into one of the other group. We apply this classification process to all 29029 premature stop codons and classify on the 665 accessions of the overlapping datasets. Due to the restriction on the smaller dataset we added a filtering step afterwards. This step removes all premature stop codons which do not have any wildtype or knock-out accession. We continue with the comparison of the gene expression differences between these two groups for each premature stop codon. 
\subsection*{The problem of two different annotations in \textit{A. thaliana}}
For the comparison of gene expression differences between wildtype and knock-out  accessions we need the transcript information for each of the accessions and to all genes, which incorporate premature stop codons. We therefore want to use the transcriptomic dataset of \textit{A. thaliana} and assign each premature stop codon a transcript level for the wildtype and knock-out group. 
The first problem we face during this step are the different annotations of the \textit{A. thaliana} genome. While the genomic sequencing reads of the 1,135 population of \textit{A. thaliana} are mapped against the Araport 11 \cite{Cheng2017} annotation, which is the most recent annotation for  \textit{A. thaliana}, the transcriptomic sequencing reads, however are mapped against one of the former annotations, which is called TAIR10 \cite{lamesch2012}. This annotation difference causes problems for correlating the mRNA levels to the right premature stop codons since genes exist in the Araport 11 annotation, which do not exist in the TAIR10 annotation since they were not known at that time. To solve this problem we remove all premature stop codons that belong to genes, which are not included into the TAIR10 annotation from our analysis. 
\subsection*{Classification of premature stop codons}
We now classify the premature stop codons, which we filtered in the previous steps into three different groups: unsignificant change of gene expression, significant decrease of gene expression and significant increase of gene expression in knock-out accessions.  Therefore we calculate the distribution of mRNA transcripts in the wildtype group and the knock-out group. Afterwards we calculate a t-test to compare the differences. 

A t-test or also called student's t-test is a statistical hypothesis test derived by W. S. Gosset 1908 \cite{student1908}. It can be used to divide if there is a significant difference between the means of two groups. The calculated p-value gives us knowledge over the strength of the evidence. In our case this means we calculate a t-test with our two groups of accessions for one of our premature stop codons. We compare the mRNA level of wildtype accessions to the distribution of mRNA levels of knock-out accessions. The calculated p-value describes how separated the groups are. A small p-value means the wildtype accessions and knock-out accessions transcript levels are separable, whereas a high p-value means that the wildtype accessions and knock-out accessions transcript levels are statistically indistinguishable.

After calculation of all p-values we introduced two significant thresholds with which we filter our list of premature stop codons to a smaller list, which includes all premature stop codons with significant differences in gene expression between the wildtype and knock-out accessions with regard to transcript levels. The first threshold is the commonly used p-value threshold of 0.05. The second threshold corrects for multiple testing. Since decision errors can accumulate under multiple testing, we need to adapt our p-value threshold to decrease the probability of false positives. To accomplish this we use the bonferroni correction\cite{dunn1961}, which calculates a new significance threshold by considering the number of tests calculated: $0.05 / \text{number\_of\_tests}$. 

By selecting the significant (0.05) and bonferroni-significant sets of premature stop codons we already classified them into two groups: the premature stop codons without significant difference in gene expression level between wildtype and knock-out accessions and the ones, who have a significant difference. We separate the significant ones further by comparing if the wildtype accessions have a higher gene expression than the knock-out accessions. These premature stop codons are classified as codons, which decrease the gene expression. We also group premature stop codons, which increase the gene expression, compared to wildtype accessions, which have a lower gene expression than the knock-out accessions and visualize examples of these three groups in \autoref{fig:Cathegories}.

\subsection*{Transformation of single-codon to gene-based premature stops}
The last step to generate a high confidential dataset is the summary of the single premature stop codons to a gene-based level. For this we just selected the bonferroni significant premature stop codons with decreased gene expression. If we only consider premature stop codons with bonferroni significant decrease in gene expression we summarize all of these premature stop codons, which occur in the same gene by increasing the groups of accessions. We summarize the knock-out group to all accessions in which at least one premature stop codon occurs and substract these from the accessions of wildtype. This step will increase the power for our further analyses since it empowers the number of knock out accesssions and decreases  the number of wildtype accessions. It also corrects for the limited resolution since we can not look at each of the premature stop codon independently on the transcript level. 

\section{Classification of premature stop codons based on the length of the remaining protein} 
For each of the 29029 premature stop codons in the population of 1,135 accessions of \textit{A. thaliana} we calculate the remaining length of the protein after the premature stop codon mutation appeared. Therefore we use the Araport 11 annotation of \textit{A. thaliana} to connect each premature stop codon with its corresponding gene. Afterwards we take the information of the start and the stop of the coding sequence (called the protein start and protein stop in the annotation) and calculate the length of the original protein. After this we calculate the remaining length for the protein when a stop codon is inserted. In the end we calculate the relative length of the protein in a way that we can access the information of how much of the protein is left after a premature stop was inserted. Two cutoffs were compared to generate high confidential datasets. The first cutoff describes premature stop codons which result in a truncated protein with less than \SI{30}{\percent} of the original protein, while the second threshold selects a high confidential dataset with an equal amount of premature stop codons compared to the bonferroni significant decreased gene expression dataset. The results are shown in \autoref{fig:Length}. 
\section{Comparison between the two subsets}
For comparing the similarities and dissimilarities between the subsets from approach one and two we analyse their overlapping premature stop codons. We first compare the subset from approach one, the bonferroni corrected subset with decreased gene expression, to  the dataset of the relative lengths with the equal amount of premature stop codons and continue with the comparison to the dataset of premature stop codons, which provoke \SI{30}{\percent} of the original proteins. We visualize the overlaps in venn diagrams shown in \autoref{fig:Venn_Diagram}. 
\section{Cooccurrence of premature stop codons}
\label{sec:Methods_Coocurrence}
To analyse the interaction of premature stop codons, we investigate the cooccurrence of pairs of premature stop codons. To accomplish this, we build all possible pairs of premature stop codons from the high confidential dataset of approach 1 (bonferroni significant decrease in gene expression). For all pairs we compute the number of accessions in which the first premature stop codon exists, in which the second premature stop codon exists and in which both premature stop codons exists simultaneously (cooccurrence). 

In order to quantify the interaction between these premature stop codon pairs, we calculate a hypergeometric test for all of them. From the result we can then conclude if the cooccurrence deviates from the statistically expected distribution. The hypergeometric test is based on the hypergeometric distribution, which can be transferred to a classical urn problem in combinatorial statistics. The urn contains a mixture of red and blue marbles. Now marbles are drawn a particular number of times without replacing them in the urn. After completing the drawing the number of drawn red marbles is evaluated. The hypergeometric test is used to determine the statistical significance of the present problem. When observing a specific number of red marbles (k) from a total number of marbles (N) with a number of red marbles in the urn (K), the calculated p-value describes how likely the event happens. In other words it describes, whether the observed number is truly random or over-represented or under-represented (Rivals 2007 \cite{rivals2007}). 

In the application of the hypergeometric test to premature stop codon interactions the total number of marbles in the urn is given by the number of accessions,  which are 665. The number of red marbles in the urn is the number of accessions with the second premature stop codons. The number of draws from the urn corresponds the number of accessions with the first premature stop codons. With these distributions we calculate a p-value for how significant the number of red marbles is separated from the expectation, which represents in our case the number of observed cooccurrence of both premature stop codons and secondly we calculate an expectation interval, which shows us the expected interval for the given urn composition. This test is calculated for all premature stop codons pairs. 

For calculating the expectation interval we need to provide a p-value threshold. We calculate all tests with two different p-value significance levels. First the commonly used 0.05 p-value threshold and second the p-value threshold corrected for multiple testing (bonferroni) as explained in \autoref{sec:Methods_Approach_Gene_Expression}. 

To select premature stop codon with a significant p-value not only the p-value threshold is needed since on genes linkage equilibrium acts. Therefore we need to apply another filter. We filter out all significant premature stop codon that are closer than 10 kbp since genes with a smaller distance are linked together and are inherited together. Because of this linkage effect they can not be treated as independent from each other\cite{Slatkin2008}. 10 kb is a commonly used threshold for correcting linkage equilibrium effects, which we also use as a distance threshold. The calculation of the distance between a pair of premature stop codons is based on the Araport11 annotation and calculates the distance between the first premature stop genes stop and the second premature stop genes start (similar to the relative length approach). 

\section{Control with synonymous mutations}
\label{sec:Methods_Control}
Our results show that we can identify pairs of premature stop codons,  which occur together more often than alone. This leads us to study a control group. An ideal control group to compare the premature stop codon cooccurrence are synonymous mutations. Therefore we repeated our interaction analysis with these control group. First we generated a control dataset of synonymous mutations, which matches the number and the allele frequency of the premature stop codons but is generated on a randomized basis.

By creating once again interaction pairs between all possibilities we calculate the cooccurrence and hypergeometric test for this control dataset. These numbers can than be compared to the premature stop codons and interpreted in the genomic context. 

\section{Control with GWAS Analysis}
To include a second control for the hypergeometric test we perform genome-wide association studies (GWAS) on all bonferroni significant interaction pairs. This analysis was not part of this master thesis and will be evaluated later. Therefore we do not discuss the results here. Nevertheless we already prepared this analysis by generating the necessary phenotypes for the GWAS calculation. 

The phenotypes are built on the basis of interaction pairs, that show bonferroni significant over-cooccurrence of these two premature stop codons. For a given pair of premature stop codons the phenotype includes all accessions, which include the first premature stop codon while the occurrence of the second premature stop codon is coded as a phenotypic value. This means that an accession, which has the first and the second premature stop codon simultaneously gets the phenotypic value of 1 and an accession, which just has the first premature stop codon receives the value of 0. For each significant over-coocurrence pair of premature stop codons this phenotype is built. 
